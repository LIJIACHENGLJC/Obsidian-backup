% Options for packages loaded elsewhere
\PassOptionsToPackage{unicode}{hyperref}
\PassOptionsToPackage{hyphens}{url}
%
\documentclass[
]{article}
\usepackage{amsmath,amssymb}
\usepackage{iftex}
\ifPDFTeX
  \usepackage[T1]{fontenc}
  \usepackage[utf8]{inputenc}
  \usepackage{textcomp} % provide euro and other symbols
\else % if luatex or xetex
  \usepackage{unicode-math} % this also loads fontspec
  \defaultfontfeatures{Scale=MatchLowercase}
  \defaultfontfeatures[\rmfamily]{Ligatures=TeX,Scale=1}
\fi
\usepackage{lmodern}
\ifPDFTeX\else
  % xetex/luatex font selection
\fi
% Use upquote if available, for straight quotes in verbatim environments
\IfFileExists{upquote.sty}{\usepackage{upquote}}{}
\IfFileExists{microtype.sty}{% use microtype if available
  \usepackage[]{microtype}
  \UseMicrotypeSet[protrusion]{basicmath} % disable protrusion for tt fonts
}{}
\makeatletter
\@ifundefined{KOMAClassName}{% if non-KOMA class
  \IfFileExists{parskip.sty}{%
    \usepackage{parskip}
  }{% else
    \setlength{\parindent}{0pt}
    \setlength{\parskip}{6pt plus 2pt minus 1pt}}
}{% if KOMA class
  \KOMAoptions{parskip=half}}
\makeatother
\usepackage{xcolor}
\setlength{\emergencystretch}{3em} % prevent overfull lines
\providecommand{\tightlist}{%
  \setlength{\itemsep}{0pt}\setlength{\parskip}{0pt}}
\setcounter{secnumdepth}{-\maxdimen} % remove section numbering
\ifLuaTeX
  \usepackage{selnolig}  % disable illegal ligatures
\fi
\IfFileExists{bookmark.sty}{\usepackage{bookmark}}{\usepackage{hyperref}}
\IfFileExists{xurl.sty}{\usepackage{xurl}}{} % add URL line breaks if available
\urlstyle{same}
\hypersetup{
  pdftitle={博士研究计划},
  hidelinks,
  pdfcreator={LaTeX via pandoc}}

\title{博士研究计划}
\author{}
\date{}

\begin{document}
\maketitle

\begin{quote}
一个好的研究计划,主要的要求体现在:问题明确,有创新价值,选题难度适中并且有可行性。所以需要分为几个部分来介绍你整体的一份好的研究计划包括以下几个部分:研究题目、研究背景、研究现状、研究目的,研究方法、进度安排,参考文献。
至关重要的是,你有机会在该研究领域传达你的研究态度,并就你项目完成的工作作出有说服力的论证。虽然研究计划书应该包括一个大纲,但它也应该作为一个有说服力的文章来处理
------
也就是说,这是一个吸引读者注意,并使他们相信你的项目重要性的机会。
\end{quote}

\subsection{问题}\label{ux95eeux9898}

\begin{enumerate}
\tightlist
\item
  选题,研究的大致方向

  \begin{enumerate}
  \tightlist
  \item
    控制类或者其他
  \end{enumerate}
\item
  研究背景
\end{enumerate}

\subsection{研究题目}\label{ux7814ux7a76ux9898ux76ee}

\begin{itemize}
\tightlist
\item
  非线性控制
\item
  最优化

  \begin{itemize}
  \tightlist
  \item
    动态优化
  \end{itemize}
\item
  系统辨识
\item
  机器人

  \begin{itemize}
  \tightlist
  \item
    环境感知
  \item
    建图
  \item
    定位
  \item
    规划
  \item
    导航
  \item
    系统控制与优化
  \end{itemize}
\end{itemize}

\subsection{选题背景及依据}\label{ux9009ux9898ux80ccux666fux53caux4f9dux636e}

\subsection{研究意义}\label{ux7814ux7a76ux610fux4e49}

\subsection{国内外研究现状及发展前景}\label{ux56fdux5185ux5916ux7814ux7a76ux73b0ux72b6ux53caux53d1ux5c55ux524dux666f}

\subsection{研究内容}\label{ux7814ux7a76ux5185ux5bb9}

\subsubsection{研究目标}\label{ux7814ux7a76ux76eeux6807}

\subsubsection{研究内容}\label{ux7814ux7a76ux5185ux5bb9-1}

\subsubsection{创新之处}\label{ux521bux65b0ux4e4bux5904}

\subsubsection{拟解决的关键问题}\label{ux62dfux89e3ux51b3ux7684ux5173ux952eux95eeux9898}

\subsection{拟采取的研究方法、技术路线}\label{ux62dfux91c7ux53d6ux7684ux7814ux7a76ux65b9ux6cd5ux6280ux672fux8defux7ebf}

{}

adsf

\end{document}
